\documentclass{ecnuthesis}
% \documentclass[printMode]{ecnuthesis}
% 模版选项:
% printMode     是否开启打印模式, 若缺省则为关闭, 反之则为开启
% 用法示例
% \documentclass[printMode]{ecnuthesis}   (开启打印模式, 适合双面打印)
% \documentclass[printMode]{ecnuthesis}   (关闭打印模式, 适合提交电子版)

\ecnuSetup {
  % 参数设置
  % 允许采用两种方式设置选项:
  %   1. style/... = ...
  %   2. style = { ... = ... } 
  % 注意事项: 
  %   1. 请勿在参数设置中出现空行
  %   2. "=" 两侧的空格将被忽略
  %   3. "/" 两侧的空格不会被忽略
  %   4. 请使用英文逗号 "," 分隔选项
  %
  % info 类用于输入论文信息
  info = {
    title = {基于微信小程序的线下课程考勤系统的设计与实现},
    % 中文标题
    %
    titleEN = {Design and Implementation of Offline Course Attendance System Based on WeChat Mini Program},
    % 英文标题
    %
    author = {张三},
    % 作者姓名
    %
    studentID = {10175101100},
    % 作者学号
    %
    department = {软件工程学院},
    % 学院名称
    %
    major = {软件工程},
    % 专业名称
    %
    supervisor = {李四},
    % 指导教师姓名
    %
    academicTitle = {教授},
    % 指导教师职称
    %
    year  = 2021,
    % 论文完成年份
    %
    month = 5,
    % 论文完成月份
    %
    keywords = {微信小程序, 课堂考勤, Java 语言, MySQL 数据库},
    % 中文关键词
    % 请使用英文逗号 "," 以分隔
    %
    keywordsEN = {WeChat Mini Program, Class attendance, Java, MySQL},
    % 英文关键词
    % 请使用英文逗号 "," 以分隔
    %
  },
  % style 类用于简单设置论文格式
  style = {
    footnote  = plain,
    % 脚注编号样式
    % 可用选项:
    %   footnote = plain|circled
    % 说明:
    %   plain     脚注的编号仅为数字
    %   circled   脚注的编号为带圆圈数字 (仅限1-10)
    %   (默认选项为 plain )
    %
    numbering = arabic,
    % 章节编号样式
    % 可用选项:
    %   numbering = arabic|alpha|chinese
    % 说明:
    %   arabic    使用数字进行编号 (即理科要求)
    %   alpha     使用字母进行编号 (即外文要求)
    %   chinese   使用汉字进行编号 (即文科要求)
    %   (默认选项为 arabic )
    %
    fontCJK = fandol,
    % 中文字体选择
    % 可用选项:
    %   fontCJK = fandol|windows|mac|default
    % 说明:
    %   fandol    使用 TeX 自带的 fandol 字体
    %   windows   使用 Windows 系统内的字体 (中易)
    %   mac       使用 MacOS 系统内的字体
    %   (默认选项为 fandol )
    %
    bibResource = {./source/thesis-ref.bib},
    % 参考文献数据源
    % 由于使用的是 biber + biblatex , 所以必须明确给出 .bib 后缀名
    %
    % logoResource = {./source/inner-cover(contains_font).eps},
    % 封面插图数据源
    % 模版已自带, 默认选项也已经设置为 ./source/inner-cover(contains_font).eps
  }
}

% 需要的宏包可以自行调用
\usepackage{mwe}

\begin{document}

% 设置前置部分编号
\frontmatter

% 中文摘要环境
\begin{abstract}
  那只敏捷的棕色狐狸跳过那只懒狗,消失得无影无踪。那只敏捷的棕色狐狸跳过那只懒狗,消失得无影无踪。那只敏捷的棕色狐狸跳过那只懒狗,消失得无影无踪。那只敏捷的棕色狐狸跳过那只懒狗,消失得无影无踪。那只敏捷的棕色狐狸跳过那只懒狗,消失得无影无踪。那只敏捷的棕色狐狸跳过那只懒狗,消失得无影无踪。
\end{abstract}

% 英文摘要环境
\begin{abstractEN}
  \blindtext
\end{abstractEN}

% 设置正文编号
\mainmatter

\chapter{绪论}
\section{背景}
\subsection{浅谈中国软件}
\subsubsection{简介}

\blindtext

\paragraph{这是一个段落}

这是段落的内容。后面跟着一个定理。

\begin{theorem}[素数定理]
  设$x \geqslant 1$, 记$\pi(x)$表示不超过$x$的素数的个数, 则当$x \to \infty$时成立
  \[\pi(x) \sim \frac{x}{\ln x}\]
\end{theorem}

\begin{proof}
  这是一段证明的内容。
  \[\pi(x) \sim \frac{x}{\ln x}\]
  命题得证。
\end{proof}

这是代码块儿:

\begin{lstlisting}[language=C++]
#include <stdio.h>
int main(int argc, char **argv) {
    printf("Hello world!\n");
    return 0;
}
\end{lstlisting}

这是行间公式:

\[
  \lim_{p\to+\infty}\int_{a}^{b}f(x)\sin{px}\,\mathrm{d}x = 0.
\]

\chapter{新的章节}

\blindtext

\blindtext

\blindtext

\blindtext

\begin{figure}
    \centering
    \includegraphics[width=.5\textwidth]{example-image}
    \bicaption{标题不能太长,否则不能居中}{English Caption}
    \label{example}
  \end{figure}

如图 \ref{example} 所示,是支持中英双语标题的图片示例。

这是脚注 \footnote{用法:\ttfamily$\backslash$footnote\{脚注内容\}} 的使用说明。

这是参考文献引用的示例 \cite{Yang_Hy200215, Zheng_wb200108}。注意,参考文献需要使用 {\ttfamily latexmk} 编译后才能使用。

% 正文后部分
\backmatter
% 导入参考文献 (需要通过 latexmk 编译后才能显示)
\PrintReference

% 附录环境
\begin{appendix}
  \blindtext
\end{appendix}

% 致谢环境
\begin{acknowledgement}
  感谢。
\end{acknowledgement}

\end{document}